\documentclass{vvsu}

\vvsuyear{2025}

\usepackage{graphicx} % для изображений
\usepackage{tabularray} % для таблиц
\usepackage{siunitx} % для обозначений (процент, градус)
\usepackage{listings} % для листингов кода

\graphicspath{ {imglabs4/} }



\author{Герцов Д.Е.}


\begin{document}

%Шапочка
\vvsuhead{\linespread{1}\selectfont{}МИНОБРНАУКИ РОССИИ\\
\vspace{10pt}Федеральное государственное бюджетное образовательное учреждение\\
высшего образования\\
\fontsize{13}{13}\selectfont{}<<ВЛАДИВОСТОКСКИЙ ГОСУДАРСТВЕННЫЙ УНИВЕРСИТЕТ>>\\
(ФГБОУ ВО <<ВВГУ>>)\\
\vspace{10pt}\fontsize{12}{12}\selectfont{}ИНСТИТУТ ИНФОРМАЦИОННЫХ ТЕХНОЛОГИЙ И АНАЛИЗА ДАННЫХ\\
КАФЕДРА ИНФОРМАЦИОННЫХ ТЕХНОЛОГИЙ И СИСТЕМ}

\title{Отчет\\по лабораторной работе №4}
\subtitle{по дисциплине\\<<Информатика и программирование>>}

% Участники работы
\member{Студент\\ гр. БИН-25-3}{Герцов Д.Е.}
\member{Ассистент\\ преподавателя}{М.В. Водяницкий}

\maketitle

\begin{addition}{Задание}
    Выполнить задания на Python и оформить отчет по стандартам ВВГУ.

    \textit{\textbf{Задание 1.}}
    Написать программу, которая определяет, как будет вести себя кондиционер. Если температура в помещении 20 градусов и выше, то кондиционер выключается, если меньше - включается. Температура должна вводится пользователем с консоли.

    \textit{\textbf{Задание 2.}}
    Год делится на четыре сезона: зима, весна, лето и осень. Написать программу, которая запрашивает у пользователя номер месяца и выводит к какому сезону этот месяц относится.

    \textit{\textbf{Задание 3.}}
    Считается, что один год, прожитый собакой, эквивалентен семи человеческим годам. При этом зачастую не учитывается, что собаки становятся абсолютно взрослыми уже к двум годам. Таким образом, многие предпочитают каждый из первых двух лет жизни собаки приравнивать к 10.5 годам человеческой жизни, а все последующие к 4.

    Написать программу, которая будет переводить собачий возраст в человеческий. Программа должна корректно обрабатывать входные данные и выводить соответствующие сообщения об ошибках

    \textit{\textbf{Задание 4.}}
    Число делиться на 6 только в случае соблюдения двух условий:

        Последняя цифра четная
        Сумма всех цифр делиться на 3

    Написать программу, которая выведет делиться ли введенное число на 6 или нет.

    \textit{\textbf{Задание 5.}}
    Написать программу, которая будет проверять пароль на надежность. Пароль считается надежным, если его длина не менее 8 символов и если он содержит:

        Заглавные буквы латиницы
        Строчные буквы латиницы
        Числа
        Специальные знаки

    В случае, если пароль не проходит по одному из условий, необходимо сообщить пользователю каким именно условиям он не удовлетворяет.

    \textit{\textbf{Задание 6.}}
    Написать программу, которая определяет, является ли введенный пользователем год високосным. Год считается високосным, если он делится на 4, но не делится на 100, либо если он делится на 400.

    \textit{\textbf{Задание 7.}}
    Задание 7
    Написать программу, которая запрашивает у пользователя три числа и выводит на экран наименьшее из них. При решении нельзя использовать встроенные функции min() и max().

    \textit{\textbf{Задание 8.}}
    В магазине проводится акция. Акция работает по следующим правилам:
    Сумма покупки 	Скидка
    до 1000 	0%
    1000–5000 	5%
    5000–10000 	10%
    более 10000 	15%

    Напишите программу, которая запрашивает сумму покупки и выводит размер скидки и итоговую сумму к оплате.

    \textit{\textbf{Задание 9.}}
    Написать программу, которая определяет время суток по введенному часу (целое число от 0 до 23).
    Время 	Период
    0–5 	Ночь
    6–11 	Утро
    12–17 	День
    18–23 	Вечер

    \textit{\textbf{Задание 10.}}
    Написать программу, которая определяет, является ли введенное число простым. Число называется простым, если оно больше 1 и делится только на 1 и само себя. Программа должна корректно обрабатывать некорректный ввод и выводить соответствующие сообщения об ошибках.

\end{addition}
% Содержание
\toc

% Глава - Выполнение работы
\section{Выполнение работы}

% Подглава - Задание 1
\subsection{Задание 1}
В данном задании создана переменная, которая хранит число. После присвоение значений, делается проверка на то больше ли 20 градусов или нет, что бы затем выключать или включать кондиционер. На рисунке 1 \ref{fig:task1} представлен полученный код.


\end{document}

