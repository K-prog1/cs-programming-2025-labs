\documentclass{vvsu}

\vvsuyear{2025}

\usepackage{graphicx} % для изображений
\usepackage{tabularray} % для таблиц
\usepackage{siunitx} % для обозначений (процент, градус)
\usepackage{listings} % для листингов кода


\graphicspath{ {../img/imglabs4} }


\author{Герцов Д.Е.}

\setmonofont{Consolas}

\makeatletter

\newcommand\language@yaml{yaml}
\expandafter\expandafter\expandafter\lstdefinelanguage
\expandafter{\language@yaml}
{
  keywords={true,false,null,y,n},
  keywordstyle=\color{darkgray},
  basicstyle=\setmainfont{Consolas}\fontsize{8}{8}\linespread{1}\selectfont,
  sensitive=false,
  comment=[l]{\#},
  morecomment=[s]{/*}{*/},
  commentstyle=\color{purple},
  stringstyle=\color{blue},
  moredelim=[l][\color{orange}]{\&},
  moredelim=[l][\color{magenta}]{*},
  moredelim=**[il][\color{red}{:}\color{blue}]{:},
  morestring=[b]',
  morestring=[b]",
  literate =    {---}{{\ProcessThreeDashes}}3
                {>}{{\textcolor{red}\textgreater}}1
                {|}{{\textcolor{red}\textbar}}1
                {\ -\ }{{\ -\ }}3,
}
\lst@AddToHook{EveryLine}{\ifx\lst@language\language@yaml\color{black}\fi}
\makeatother

\lstdefinelanguage{json}{
    basicstyle=\fontsize{8}{8}\linespread{1}\selectfont\ttfamily,
    sensitive=false,
    stringstyle=\color{blue},
    string=[s]{":\ "}{"},
    literate=
        *{0}{{{\color{red}0}}}{1}
         {1}{{{\color{red}1}}}{1}
         {2}{{{\color{red}2}}}{1}
         {3}{{{\color{red}3}}}{1}
         {4}{{{\color{red}4}}}{1}
         {5}{{{\color{red}5}}}{1}
         {6}{{{\color{red}6}}}{1}
         {7}{{{\color{red}7}}}{1}
         {8}{{{\color{red}8}}}{1}
         {9}{{{\color{red}9}}}{1}
}

\definecolor{codegray}{rgb}{0.5,0.5,0.5}
\definecolor{backcolour}{rgb}{0.95,0.95,0.92}
\lstdefinestyle{codestylelst}{
    backgroundcolor=\color{backcolour},
    numberstyle=\color{codegray}\ttfamily,
    breakatwhitespace=false,
    breaklines=true,
    captionpos=b,
    keepspaces=true,
    numbers=left,
    numbersep=5pt,
    showspaces=false,
    showstringspaces=false,
    showtabs=false,
    tabsize=2
}
\lstset{style=codestylelst}

\begin{document}

%Шапочка
\vvsuhead{\linespread{1}\selectfont{}МИНОБРНАУКИ РОССИИ\\
\vspace{10pt}Федеральное государственное бюджетное образовательное учреждение\\
высшего образования\\
\fontsize{13}{13}\selectfont{}<<ВЛАДИВОСТОКСКИЙ ГОСУДАРСТВЕННЫЙ УНИВЕРСИТЕТ>>\\
(ФГБОУ ВО <<ВВГУ>>)\\
\vspace{10pt}\fontsize{12}{12}\selectfont{}ИНСТИТУТ ИНФОРМАЦИОННЫХ ТЕХНОЛОГИЙ И АНАЛИЗА ДАННЫХ\\
КАФЕДРА ИНФОРМАЦИОННЫХ ТЕХНОЛОГИЙ И СИСТЕМ}

\title{Отчет\\по лабораторной работе №4}
\subtitle{по дисциплине\\<<Информатика и программирование>>}

% Участники работы
\member{Студент\\ гр. БИН-25-3}{Герцов Д.Е.}
\member{Ассистент\\ преподавателя}{М.В. Водяницкий}

\maketitle

\begin{addition}{Задание}
    Выполнить задания на Python и оформить отчет по стандартам ВВГУ.

    \textit{\textbf{Задание 1.}}
    Написать программу, которая определяет, как будет вести себя кондиционер. Если температура в помещении 20 градусов и выше, то кондиционер выключается, если меньше - включается. Температура должна вводится пользователем с консоли.

    \textit{\textbf{Задание 2.}}
    Год делится на четыре сезона: зима, весна, лето и осень. Написать программу, которая запрашивает у пользователя номер месяца и выводит к какому сезону этот месяц относится.

    \textit{\textbf{Задание 3.}}
    Считается, что один год, прожитый собакой, эквивалентен семи человеческим годам. При этом зачастую не учитывается, что собаки становятся абсолютно взрослыми уже к двум годам. Таким образом, многие предпочитают каждый из первых двух лет жизни собаки приравнивать к 10.5 годам человеческой жизни, а все последующие к 4.

    Написать программу, которая будет переводить собачий возраст в человеческий. Программа должна корректно обрабатывать входные данные и выводить соответствующие сообщения об ошибках

    \textit{\textbf{Задание 4.}}
    Число делиться на 6 только в случае соблюдения двух условий:

        Последняя цифра четная
        Сумма всех цифр делиться на 3

    Написать программу, которая выведет делиться ли введенное число на 6 или нет.

    \textit{\textbf{Задание 5.}}
    Написать программу, которая будет проверять пароль на надежность. Пароль считается надежным, если его длина не менее 8 символов и если он содержит:

        Заглавные буквы латиницы
        Строчные буквы латиницы
        Числа
        Специальные знаки

    В случае, если пароль не проходит по одному из условий, необходимо сообщить пользователю каким именно условиям он не удовлетворяет.

    \textit{\textbf{Задание 6.}}
    Написать программу, которая определяет, является ли введенный пользователем год високосным. Год считается високосным, если он делится на 4, но не делится на 100, либо если он делится на 400.

    \textit{\textbf{Задание 7.}}
    Задание 7
    Написать программу, которая запрашивает у пользователя три числа и выводит на экран наименьшее из них. При решении нельзя использовать встроенные функции min() и max().

    \textit{\textbf{Задание 8.}}
    В магазине проводится акция. Акция работает по следующим правилам:
    Сумма покупки 	Скидка
    до 1000 	0%
    1000–5000 	5%
    5000–10000 	10%
    более 10000 	15%

    Напишите программу, которая запрашивает сумму покупки и выводит размер скидки и итоговую сумму к оплате.

    \textit{\textbf{Задание 9.}}
    Написать программу, которая определяет время суток по введенному часу (целое число от 0 до 23).
    Время 	Период
    0–5 	Ночь
    6–11 	Утро
    12–17 	День
    18–23 	Вечер

    \textit{\textbf{Задание 10.}}
    Написать программу, которая определяет, является ли введенное число простым. Число называется простым, если оно больше 1 и делится только на 1 и само себя. Программа должна корректно обрабатывать некорректный ввод и выводить соответствующие сообщения об ошибках.

\end{addition}
% Содержание
\toc

% Глава - Выполнение работы
\section{Выполнение работы}

% Подглава - Задание 1

\subsection{Задание 1}
В данном задании создана переменная, которая хранит число. После присвоение значений, делается проверка на то больше ли 20 градусов или нет, что бы затем выключать или включать кондиционер. На рисунке \ref{fig:task1} представлен полученный код.

\begin{vvsu_figure}{Листинг программы для задания 1}{fig:task1}
    \begin{minipage}{.75\textwidth}
        \lstinputlisting[language=Python,basicstyle=\fontsize{10}{10}\linespread{1}\selectfont\ttfamily]{../ImgForLaTexReport/labs4python/task1.py}
    \end{minipage}
\end{vvsu_figure}

\begin{vvsu_list}
    \item Создаем фунцию для вызова
    \item Переменную gradus,которую вводим целое число через консоль
    \item Делаем проверку условий, если больше 20 то выключается, если меньше -работает 
\end{vvsu_list}

После выполнение программы взависимости от введеного числа, будет получен нужный результат.



\subsection{Задание 2}
В данном задании создана переменная в которую можно ввести число. После присвоения значений делается проверка через выяснения промежутка число от 1 до 12, где за тем определяется время года. На рисунке \ref{fig:task2} представлен полученный код.

\begin{vvsu_figure}{Листинг программы для задания 2}{fig:task2}
    \begin{minipage}{.75\textwidth}
        \lstinputlisting[language=Python,basicstyle=\fontsize{10}{10}\linespread{1}\selectfont\ttfamily]{../ImgForLaTexReport/labs4python/task2.py}
    \end{minipage}
\end{vvsu_figure}

\begin{vvsu_list}
    \item Создаем фунцию для вызова
    \item Переменную month,которую вводим целое число через консоль
    \item Делаем проверку условий, если 0< month < 13, то число принадлежит нужому промежутку
    \item Далается проверка, какому промежутку принадлежит число и выводится результат в консоль
\end{vvsu_list}
После выполнение программы взависимости от введеного числа, будет получено соответствующие время года

\subsection{Задание 3}
В данном задании создана переменная в которую можно ввести число и сделана проверка на то, введено число или другой символ. Затем создана переменная human, котоая хранит в себе данные 10.5. Делается проверка в каком промежутке стоит переменная, которую мы ввели. Если переменная равна 1, то пишется 10,5. Если переменная больше 1, идет цикл, который будет длиться, пока введеное чисо не будет равняться 1. В цикле будет выполняться два действия: отнятие единицы от переменной, которой мы ввели число и прибовляться 4 к переменной human. После выполнения будет выводиться в консоль переменная human и отниматься 4. На рисунке \ref{fig:task3} представлен полученный код.

\begin{vvsu_figure}{Листинг программы для задания 3}{fig:task3}
    \begin{minipage}{.75\textwidth}
        \lstinputlisting[language=Python,basicstyle=\fontsize{10}{10}\linespread{1}\selectfont\ttfamily]{../ImgForLaTexReport/labs4python/task3.py}
    \end{minipage}
\end{vvsu_figure}

\begin{vvsu_list}
    \item Создаем фунцию
    \item Переменную sobak, которую вводим как целое число через консоль
    \item Обрабатываем исключение ValueError
    \item Проверяем условие: если 0<sobak<23, вычисляем возраст по формуле
    \item Если sobak <0 или sobak > 22, выводим соответствующее сообщение
\end{vvsu_list}

После выполнение программы взависимости от введеного числа, будет получен соотвествующий ответ

\subsection{Задание 4}
В данном задании создана переменная number, в которую можно ввести число. Затем создана переменная summa, которая вычисляет сумму цифр введенного числа с помощью функций list(), map() и sum(). Далее выполняется проверка условий: если сумма цифр числа делится на 3 без остатка И последняя цифра числа является четной, то выводится сообщение "Число делится на 3!". В противном случае выводится сообщение "Не делится на 3". На рисунке \ref{fig:task4} представлен полученный код.

\begin{vvsu_figure}{Листинг программы для задания 4}{fig:task4}
    \begin{minipage}{.75\textwidth}
        \lstinputlisting[language=Python,basicstyle=\fontsize{10}{10}\linespread{1}\selectfont\ttfamily]{../ImgForLaTexReport/labs4python/task4.py}
    \end{minipage}
\end{vvsu_figure}

\begin{vvsu_list}
\item Создаем функцию 
\item Переменную number, которую вводим через консоль
\item Создаем переменную summa, которая вычисляет сумму цифр числа
\item Проверяем условие: если сумма цифр делится на 3 И последняя цифра числа четная
\item Если условие выполняется, выводим "Число делится на 3!"
\item Если условие не выполняется, выводим "Не делится на 3"
\end{vvsu_list}


После выполнение программы взависимости от введеного числа, будет получен соответствующий ответ

\subsection{Задание 5}
В данном задании создана переменная password, в которую можно ввести пароль. Затем созданы переменные для проверки различных критериев пароля: наличие заглавных букв, строчных букв, цифр, специальных символов и достаточной длины. Создан список для хранения выявленных проблем. Выполняется проверка длины пароля - если меньше 8 символов, добавляется соответствующее сообщение. Затем в цикле проверяется каждый символ пароля на соответствие критериям. Если все условия выполняются, выводится сообщение о надежности пароля. В противном случае формируется список отсутствующих элементов и выводится соответствующее сообщение. На рисунке \ref{fig:task5} представлен полученный код.
\begin{vvsu_figure}{Листинг программы для задания 5}{fig:task5}
    \begin{minipage}{.75\textwidth}
        \lstinputlisting[language=Python,basicstyle=\fontsize{10}{10}\linespread{1}\selectfont\ttfamily]{../ImgForLaTexReport/labs4python/task5.py}
    \end{minipage}
\end{vvsu_figure}

\begin{vvsu_list}
\item Создаем функцию
\item Переменную password, которую вводим через консоль
\item Создаем список для хранения проблем пароля
\item Определяем строку  со специальными символами
\item Инициализируем переменные для проверки критериев пароля
\item Проверяем длину пароля (минимум 8 символов)
\item В цикле проверяем каждый символ на соответствие критериям
\item Если все условия выполнены, выводим "Пароль надежный!"
\item Иначе формируем список проблем и выводим сообщение
\end{vvsu_list}


После выполнение программы взависимости от введеного числа, будет получен соответствующий ответ

\subsection{Задание 6}
В данном задании создана переменная year, в которую можно ввести год для проверки на високосность. Выполняется обработка исключений для корректного ввода числа. Для определения високосного года используется условие: год должен делиться на 4 без остатка, но не делиться на 100, либо делиться на 400 без остатка. Если условие выполняется, выводится сообщение о том, что год високосный. В противном случае выводится сообщение, что год не високосный. При вводе некорректных данных выводится соответствующее предупреждение. На рисунке \ref{fig:task6} представлен полученный код.
\begin{vvsu_figure}{Листинг программы для задания 6}{fig:task6}
    \begin{minipage}{.75\textwidth}
        \lstinputlisting[language=Python,basicstyle=\fontsize{10}{10}\linespread{1}\selectfont\ttfamily]{../ImgForLaTexReport/labs4python/task6.py}
    \end{minipage}
\end{vvsu_figure}

\begin{vvsu_list}
\item Создаем функцию 
\item Переменную year, которую вводим как целое число через консоль
\item Обрабатываем исключения для корректного ввода
\item Проверяем условие високосности года: делится на 4, но не на 100, либо делится на 400
\item Если условие выполняется, выводим сообщение о високосном годе
\item Если условие не выполняется, выводим сообщение о невисокосном годе
\item При ошибке ввода выводим сообщение "Введи цифры, а не чепуху"
\end{vvsu_list}


После выполнение программы взависимости от введеного числа, будет получен соответствующий ответ

\subsection{Задание 7}
В данном задании создана переменная number, в которую можно ввести трехзначное число. Выполняется проверка длины введенного числа - должно быть равно 3 цифрам. Если условие выполняется, число преобразуется в список цифр. Затем выполняется поиск минимальной цифры среди трех введенных цифр путем попарного сравнения. Выводится наименьшая цифра из трехзначного числа. Если введено число, содержащее не три цифры, выводится сообщение об ошибке. На рисунке \ref{fig:task7} представлен полученный код.
\begin{vvsu_figure}{Листинг программы для задания 7}{fig:task7}
    \begin{minipage}{.75\textwidth}
        \lstinputlisting[language=Python,basicstyle=\fontsize{10}{10}\linespread{1}\selectfont\ttfamily]{../ImgForLaTexReport/labs4python/task7.py}
    \end{minipage}
\end{vvsu_figure}

\begin{vvsu_list}
\item Создаем функцию
\item Переменную number, которую вводим как целое число через консоль
\item Проверяем длину числа - должно быть равно 3 цифрам
\item Если условие выполняется, преобразуем число в список цифр
\item Сравниваем цифры попарно для нахождения минимальной
\item Выводим наименьшую цифру из трехзначного числа
\item Если введено не трехзначное число, выводим сообщение об ошибке
\end{vvsu_list}


После выполнение программы взависимости от введеного числа, будет получен соответствующий ответ

\subsection{Задание 8}
В данном задании создана переменная number, в которую можно ввести трехзначное число. Выполняется проверка длины введенного числа - должно быть равно 3 цифрам. Если условие выполняется, число преобразуется в список цифр. Затем выполняется поиск минимальной цифры среди трех введенных цифр путем попарного сравнения. Выводится наименьшая цифра из трехзначного числа. Если введено число, содержащее не три цифры, выводится сообщение об ошибке. На рисунке \ref{fig:task8} представлен полученный код.
\begin{vvsu_figure}{Листинг программы для задания 8}{fig:task8}
    \begin{minipage}{.75\textwidth}
        \lstinputlisting[language=Python,basicstyle=\fontsize{10}{10}\linespread{1}\selectfont\ttfamily]{../ImgForLaTexReport/labs4python/task8.py}
    \end{minipage}
\end{vvsu_figure}

\begin{vvsu_list}
\item Создаем функцию
\item Переменную sum, которую вводим как целое число через консоль
\item Проверяем сумму товара и определяем соответствующий диапазон скидки
\item Если сумма < 1000: скидка 0%
\item Если сумма от 1000 до 5000: скидка 5%
\item Если сумма от 5000 до 10000: скидка 10%
\item Если сумма ≥ 10000: скидка 15%
\item Для каждого случая выводим размер скидки и итоговую сумму
\end{vvsu_list}


После выполнение программы взависимости от введеного числа, будет получен соответствующий ответ

\subsection{Задание 9}
В данном задании создана переменная sum, в которую можно ввести время в часах. Выполняется проверка введенного времени и определяется соответствующая часть суток. Если время от 0 до 5 часов, выводится "Ночь". При времени от 6 до 11 часов выводится "Утро". При времени от 12 до 17 часов выводится "День". При времени от 18 до 23 часов выводится "Вечер". Если введено число больше 23 часов, выводится сообщение об ошибке. На рисунке \ref{fig:task9} представлен полученный код.
\begin{vvsu_figure}{Листинг программы для задания 9}{fig:task9}
    \begin{minipage}{.75\textwidth}
        \lstinputlisting[language=Python,basicstyle=\fontsize{10}{10}\linespread{1}\selectfont\ttfamily]{../ImgForLaTexReport/labs4python/task9.py}
    \end{minipage}
\end{vvsu_figure}

\begin{vvsu_list}
\item Создаем функцию
\item Переменную sum, которую вводим как целое число через консоль
\item Проверяем значение времени и определяем часть суток
\item Если время 0-5 часов: выводим "Ночь"
\item Если время 6-11 часов: выводим "Утро"
\item Если время 12-17 часов: выводим "День"
\item Если время 18-23 часов: выводим "Вечер"
\item Если время > 23 часов: выводим сообщение об ошибке
\end{vvsu_list}

После выполнение программы взависимости от введеного числа, будет получен соответствующий ответ

\subsection{Задание 10}
В данном задании создана переменная chislo, в которую можно ввести число для проверки на простоту. Инициализируется переменная a со значением 0 для подсчета делителей. Выполняется проверка числа в диапазоне от 2 до 9 включительно. Если число не делится без остатка на текущий делитель из диапазона, счетчик a увеличивается на 1. Если после проверки всех делителей счетчик a равен 8 (число не делится ни на один из делителей от 2 до 9), выводится сообщение о том, что число простое. В противном случае выводится сообщение, что число составное. На рисунке \ref{fig:task10} представлен полученный код.
\begin{vvsu_figure}{Листинг программы для задания 10}{fig:task10}
    \begin{minipage}{.75\textwidth}
        \lstinputlisting[language=Python,basicstyle=\fontsize{10}{10}\linespread{1}\selectfont\ttfamily]{../ImgForLaTexReport/labs4python/task10.py}
    \end{minipage}
\end{vvsu_figure}

\begin{vvsu_list}
\item Создаем функцию 
\item Переменную chislo, которую вводим как целое число через консоль
\item Инициализируем переменную a для подсчета делителей
\item Проверяем делимость числа на все значения от 2 до 9
\item Если число не делится на делитель, увеличиваем счетчик a
\item Если счетчик a равен 8, выводим "простое число"
\item Иначе выводим "составное число"
\end{vvsu_list}

После выполнения программы в зависимости от введенного числа, будет получен ответ о том, является ли число простым или составным.
\end{document}
