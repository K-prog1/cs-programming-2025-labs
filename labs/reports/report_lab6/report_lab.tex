\documentclass{vvsu}

\vvsuyear{2025}

\usepackage{graphicx} % для изображений
\usepackage{tabularray} % для таблиц
\usepackage{siunitx} % для обозначений (процент, градус)
\usepackage{listings} % для листингов кода

\graphicspath{ {../ImgForLaTexReport/labs6python} }

\author{Герцов Д.Е.}

\input{listing_styles.tex}

\begin{document}

%Шапочка
\vvsuhead{\linespread{1}\selectfont{}МИНОБРНАУКИ РОССИИ\\
\vspace{10pt}Федеральное государственное бюджетное образовательное учреждение\\
высшего образования\\
\fontsize{13}{13}\selectfont{}<<ВЛАДИВОСТОКСКИЙ ГОСУДАРСТВЕННЫЙ УНИВЕРСИТЕТ>>\\
(ФГБОУ ВО <<ВВГУ>>)\\
\vspace{10pt}\fontsize{12}{12}\selectfont{}ИНСТИТУТ ИНФОРМАЦИОННЫХ ТЕХНОЛОГИЙ И АНАЛИЗА ДАННЫХ\\
КАФЕДРА ИНФОРМАЦИОННЫХ ТЕХНОЛОГИЙ И СИСТЕМ}

\title{Отчет\\по лабораторной работе №6}
\subtitle{по дисциплине\\<<Информатика и программирование>>}

% Участники работы
\member{Студент\\ гр. БИН-25-3}{Герцов Д.Е.}
\member{Ассистент\\ преподавателя}{М.В. Водяницкий}

\maketitle

\begin{addition}{Задание}
    Выполнить задания на Python и оформить отчет по стандартам ВВГУ.

    \textit{\textbf{Задание 1.}}
    Написать функцию, которая конвертирует время из одной величины в другую. На вход подаются: число (величина времени), исходная единица измерения, единица измерения, в которую нужно перевести. Функция должна вернуть конвертированное значение.

    \textit{\textbf{Задание 2.}}
    Пользователь делает вклад в банке в размере \(a\) рублей сроком на \(n\) лет. Процент по вкладу зависит от суммы и срока. Необходимо рассчитать прибыль по сложным процентам (без учёта первоначальной суммы).

    \textit{\textbf{Задание 3.}}
    Написать функцию для вывода всех простых чисел в заданном диапазоне. Учитывать некорректные данные и отсутствие простых чисел.

    \textit{\textbf{Задание 4.}}
    Реализовать функцию сложения двух квадратных матриц. Складывать можно только матрицы размера строго больше 2. При нарушении условий — сообщение об ошибке.

    \textit{\textbf{Задание 5.}}
    Написать функцию, определяющую, является ли строка палиндромом (игнорируя регистр, пробелы и знаки препинания).

\end{addition}

\toc

\section{Выполнение работы}

\subsection{Задание 1}
Реализована функция перевода времени между единицами измерения: часы (`h`) и минуты (`m`). Ввод осуществляется в формате «значение\_единица целевая\_единица». Поддерживаются преобразования: часы → минуты и минуты → часы. На рисунке \ref{fig:task1} представлен код программы.

\begin{vvsu_figure}{Листинг программы для задания 1}{fig:task1}
    \begin{minipage}{.75\textwidth}
        \lstinputlisting[language=Python,basicstyle=\fontsize{10}{10}\linespread{1}\selectfont\ttfamily]{../../ImgForLaTexReport/labs6python/task2.py}
    \end{minipage}
\end{vvsu_figure}

\begin{vvsu_list}
    \item Ввод строки с параметрами
    \item Разделение на исходное значение и целевую единицу
    \item Проверка типа исходной единицы (<<h>> или <<m>>)
    \item Выполнение перевода с помощью арифметических операций
    \item Вывод результата
\end{vvsu_list}
После выполнение программы взависимости от введеного числа, будет получен нужный результат
\subsection{Задание 2}
Реализована система расчёта банковского вклада с учётом базовой ставки (зависит от срока) и бонусной ставки (зависит от суммы). Используется сложный процент: каждый год процент начисляется на текущую сумму. На рисунке \ref{fig:task2} представлен код программы.

\begin{vvsu_figure}{Листинг программы для задания 2}{fig:task2}
    \begin{minipage}{.75\textwidth}
        \lstinputlisting[language=Python,basicstyle=\fontsize{10}{10}\linespread{1}\selectfont\ttfamily]{../../ImgForLaTexReport/labs6python/task2.py}
    \end{minipage}
\end{vvsu_figure}

\begin{vvsu_list}
    \item Ввод суммы вклада и срока
    \item Проверка минимального вклада (30 000 руб.)
    \item Расчёт бонусной ставки (по 0.3\% за каждые 10 000 руб., максимум 5\%)
    \item Определение базовой ставки по сроку (3\%, 5\% или 2\%)
    \item Циклический расчёт итоговой суммы по сложному проценту
    \item Вывод прибыли (итоговая сумма минус первоначальный вклад)
\end{vvsu_list}
После выполнение программы взависимости от введеного числа, будет получен нужный результат
\subsection{Задание 3}
Реализована функция поиска всех простых чисел в заданном диапазоне. Простое число — натуральное число больше 1, имеющее ровно два делителя. Диапазон вводится как два целых числа. На рисунке \ref{fig:task3} представлен код программы.

\begin{vvsu_figure}{Листинг программы для задания 3}{fig:task3}
    \begin{minipage}{.75\textwidth}
        \lstinputlisting[language=Python,basicstyle=\fontsize{10}{10}\linespread{1}\selectfont\ttfamily]{../../ImgForLaTexReport/labs6python/task3.py}
    \end{minipage}
\end{vvsu_figure}

\begin{vvsu_list}
    \item Перебор всех чисел в диапазоне [start, end].
    \item Сбор простых чисел в список.
    \item Возврат результата (список или пустой список при отсутствии простых).
    \item Вывод результата.
\end{vvsu_list}
После выполнение программы взависимости от введеного числа, будет получен нужный результат
\subsection{Задание 4}
Реализована функция сложения двух квадратных матриц. Программа проверяет, что размер матрицы больше 2, и что обе матрицы имеют корректный ввод. На рисунке \ref{fig:task4} представлен код программы.

\begin{vvsu_figure}{Листинг программы для задания 4}{fig:task4}
    \begin{minipage}{.75\textwidth}
        \lstinputlisting[language=Python,basicstyle=\fontsize{10}{10}\linespread{1}\selectfont\ttfamily]{../../ImgForLaTexReport/labs6python/task4.py}
    \end{minipage}
\end{vvsu_figure}

\begin{vvsu_list}
    \item Ввод размера матрицы n
    \item Проверка условия n > 2
    \item Чтение элементов первой и второй матриц построчно
    \item Проверка корректности количества элементов в строках
    \item Поэлементное сложение матриц
    \item Вывод результирующей матрицы или сообщения об ошибке
\end{vvsu_list}
После выполнение программы взависимости от введеного числа, будет получен нужный результат
\subsection{Задание 5}
Реализована функция проверки строки на палиндром. При проверке игнорируются регистр, пробелы, знаки препинания и любые не-буквенно-цифровые символы. На рисунке \ref{fig:task5} представлен код программы.

\begin{vvsu_figure}{Листинг программы для задания 5}{fig:task5}
    \begin{minipage}{.75\textwidth}
        \lstinputlisting[language=Python,basicstyle=\fontsize{10}{10}\linespread{1}\selectfont\ttfamily]{../../ImgForLaTexReport/labs6python/task5.py}
    \end{minipage}
\end{vvsu_figure}

\begin{vvsu_list}
    \item Ввод строки от пользователя
    \item Фильтрация строки: остаются только буквы и цифры
    \item Приведение к верхнему регистру
    \item Сравнение чистой строки с её реверсом
    \item Вывод результата: «Все верно!» или «Все плохо!»
\end{vvsu_list}
После выполнение программы взависимости от введеного числа, будет получен нужный результат
\end{document}