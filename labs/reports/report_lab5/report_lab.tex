\documentclass{vvsu}

\vvsuyear{2025}

\usepackage{graphicx} % для изображений
\usepackage{tabularray} % для таблиц
\usepackage{siunitx} % для обозначений (процент, градус)
\usepackage{listings} % для листингов кода

\graphicspath{ {../ImgForLaTexReport/labs5python} }

\author{Герцов Д.Е.}

\input{listing_styles.tex}

\begin{document}

%Шапочка
\vvsuhead{\linespread{1}\selectfont{}МИНОБРНАУКИ РОССИИ\\
\vspace{10pt}Федеральное государственное бюджетное образовательное учреждение\\
высшего образования\\
\fontsize{13}{13}\selectfont{}<<ВЛАДИВОСТОКСКИЙ ГОСУДАРСТВЕННЫЙ УНИВЕРСИТЕТ>>\\
(ФГБОУ ВО <<ВВГУ>>)\\
\vspace{10pt}\fontsize{12}{12}\selectfont{}ИНСТИТУТ ИНФОРМАЦИОННЫХ ТЕХНОЛОГИЙ И АНАЛИЗА ДАННЫХ\\
КАФЕДРА ИНФОРМАЦИОННЫХ ТЕХНОЛОГИЙ И СИСТЕМ}

\title{Отчет\\по лабораторной работе №5}
\subtitle{по дисциплине\\<<Информатика и программирование>>}

% Участники работы
\member{Студент\\ гр. БИН-25-3}{Герцов Д.Е.}
\member{Ассистент\\ преподавателя}{М.В. Водяницкий}

\maketitle

\begin{addition}{Задание}
    Выполнить задания на Python и оформить отчет по стандартам ВВГУ.

    \textit{\textbf{Задание 1.}}
    Дан список из 10 различных целых чисел. Необходимо найти в нем число 3 и заменить на 30.

    \textit{\textbf{Задание 2.}}
    Дан список из 5 целых чисел. Необходимо превратить его в список квадратов этих чисел.

    \textit{\textbf{Задание 3.}}
    Имеется список различных целых чисел. Программа должна найти наибольшее из чисел списка и разделить его на длину списка.

    \textit{\textbf{Задание 4.}}
    Имеется кортеж из нескольких произвольных элементов. Необходимо этот кортеж отсортировать. Если хотя бы один элемент не является числом, то кортеж остается неизменным.

    \textit{\textbf{Задание 5.}}
    Имеется словарь товаров в магазине. Необходимо найти товар с минимальной и максимальной ценой.

    \textit{\textbf{Задание 6.}}
    Имеется список произвольных элементов. Необходимо на основе этого списка создать словарь, где каждый элемент списка будет и ключом, и значением.
    \textit{\textbf{Задание 7.}}
    Имеется словарь перевода английских слов на русский, где ключ английского слово, значение - русского. Необходимо реализовать программу которая получает на ввод русское слово и результатом выдает перевод на английский.

    \textit{\textbf{Задание 8.}}
    
    Реализовать игру Камень-Ножницы-Бумага-Ящерица-Спок. Программа должна запрашивать у пользователя ввод одного из вариантов. Второй вариант случайно генерирует сама программа и возвращает победителя.

    Правила игры следующие:

    -- Ножницы режут бумагу
    -- Бумага покрывает камень
    -- Камень давит ящерицу
    -- Ящерица отравляет Спока
    -- Спок ломает ножницы
    -- Ножницы обезглавливают ящерицу
    -- Ящерица съедает бумагу
    -- Бумага подставляет Спока
    -- Спок испаряет камень
    -- Камень разбивает ножницы

    \textit{\textbf{Задание 9.}}

    Дан список слов - например:

    `["яблоко", "груша", "банан", "киви", "апельсин", "ананас"]`

    Необходимо создать новый словарь, где:

    - Ключом будет первая буква слова
    - Значением - список всех слов, начинающихся с этой буквы

    Пример результата:
    {'я': ['яблоко'], 'г': ['груша'], 'б': ['банан'], 'к': ['киви'], 'а': ['апельсин', 'ананас']}

    \textit{\textbf{Задание 10.}}
    Написать программу, которая определяет, является ли введенное число простым. Число называется простым, если оно больше 1 и делится только на 1 и само себя. Программа должна корректно обрабатывать некорректный ввод и выводить соответствующие сообщения об ошибках.

\end{addition}

\toc

\section{Выполнение работы}

% Подглава - Задание 1

\subsection{Задание 1}
Функция изменяет оригинальный список(работает <<in-place>> ). Щаменяется только первое вхождение числа 3. Используется прямой доступ по индексу для модификации элементов. На рисунке \ref{fig:task1} представлен полученный код.

\begin{vvsu_figure}{Листинг программы для задания 1}{fig:task1}
    \begin{minipage}{.75\textwidth}
        \lstinputlisting[language=Python,basicstyle=\fontsize{10}{10}\linespread{1}\selectfont\ttfamily]{../../ImgForLaTexReport/labs5python/task1.py}
    \end{minipage}
\end{vvsu_figure}

\begin{vvsu_list}
    \item Создание списка
    \item Цикл по индексам
    \item Проверка и замена
    \item Вывадение результата
\end{vvsu_list}



\subsection{Задание 2}

 Создаем список чисел от 1 до 5. Проходим по всем индексам списка. Каждый элемент возводим в квадрат. Функция ничего не возращает и не выходит результат. На рисунке \ref{fig:task2} представлен полученный код.

\begin{vvsu_figure}{Листинг программы для задания 2}{fig:task2}
    \begin{minipage}{.75\textwidth}
        \lstinputlisting[language=Python,basicstyle=\fontsize{10}{10}\linespread{1}\selectfont\ttfamily]{../../ImgForLaTexReport/labs5python/task2.py}
    \end{minipage}
\end{vvsu_figure}

\begin{vvsu_list}
    \item Создание списка
    \item Цикл по индексам
    \item Возведение в квадрат
    \item Конечный результат изменяется
\end{vvsu_list}

\subsection{Задание 3}
Создаем список из 7 чисел. Выводим результат деления максимального элемента на длину списка. На рисунке \ref{fig:task3} представлен полученный код.

\begin{vvsu_figure}{Листинг программы для задания 3}{fig:task3}
    \begin{minipage}{.75\textwidth}
        \lstinputlisting[language=Python,basicstyle=\fontsize{10}{10}\linespread{1}\selectfont\ttfamily]{../../ImgForLaTexReport/labs5python/task3.py}
    \end{minipage}
\end{vvsu_figure}

\begin{vvsu_list}
    \item Создание списка
    \item Вычесление максимального элемента
    \item Вычесление длины списка
    \item Деление
\end{vvsu_list}

После выполнение программы взависимости от введеного числа, будет получен соотвествующий ответ

\subsection{Задание 4}
Создаем счетчик, создаем кортеж, перебираем элементы, проверяем тип, считаем числа, сравниваем счетчик с длиной, выводим результат. На рисунке \ref{fig:task4} представлен полученный код.
\begin{vvsu_figure}{Листинг программы для задания 4}{fig:task4}
    \begin{minipage}{.75\textwidth}
        \lstinputlisting[language=Python,basicstyle=\fontsize{10}{10}\linespread{1}\selectfont\ttfamily]{../../ImgForLaTexReport/labs5python/task4.py}
    \end{minipage}
\end{vvsu_figure}

\begin{vvsu_list}
    \item Создание счетчика и кортежа (1, "2", 3, "попа", 66, 43)
    \item Перебор каждого элемента кортежа
    \item Проверка типа элемента (int или нет)
    \item Подсчет количества целых чисел (4 из 6)
    \item Сравнение счетчика с длиной кортежа
    \item Вывод <<В кортеже не только числа>>
\end{vvsu_list}

\subsection{Задание 5}
   Создаем словарь, находим товар с максимальной ценой, находим товар с минимальной ценой, выводим результаты. На рисунке \ref{fig:task5} представлен полученный код.
\begin{vvsu_figure}{Листинг программы для задания 5}{fig:task5}
    \begin{minipage}{.75\textwidth}
        \lstinputlisting[language=Python,basicstyle=\fontsize{10}{10}\linespread{1}\selectfont\ttfamily]{../../ImgForLaTexReport/labs5python/task5.py}
    \end{minipage}
\end{vvsu_figure}

\begin{vvsu_list}
    \item Создание словаря товаров и цен
    \item Поиск товара с максимальной ценой (Мясо козла -- 9999)
    \item Поиск товара с минимальной ценой (Пепси-кола -- 1)
    \item Вывод самого дорогого товара
    \item Вывод самого дешевого товара
\end{vvsu_list}

\subsection{Задание 6}
 Создаем список, создаем пустой словарь, перебираем элементы списка, добавляем каждый элемент как ключ и значение. На рисунке \ref{fig:task6} представлен полученный код.
\begin{vvsu_figure}{Листинг программы для задания 6}{fig:task6}
    \begin{minipage}{.75\textwidth}
        \lstinputlisting[language=Python,basicstyle=\fontsize{10}{10}\linespread{1}\selectfont\ttfamily]{../../ImgForLaTexReport/labs5python/task6.py}
    \end{minipage}
\end{vvsu_figure}

\begin{vvsu_list}
    \item Создание списка [<<propan>>, <<клертон>>, <<флиртон>>, 123, 444, 323]
    \item Создание пустого словаря
    \item Перебор каждого элемента списка
    \item Добавление элемента как ключа и значения в словарь
    \item Формирование словаря {элемент: элемент}
    \item Вывод готового словаря
\end{vvsu_list}

\subsection{Задание 7}
Создаем словарь переводов, получаем русское слово от пользователя, ищем английский перевод, выводим результат или сообщение об ошибке. На рисунке \ref{fig:task7} представлен полученный код.

\begin{vvsu_figure}{Листинг программы для задания 7}{fig:task7}
    \begin{minipage}{.75\textwidth}
        \lstinputlisting[language=Python,basicstyle=\fontsize{10}{10}\linespread{1}\selectfont\ttfamily]{../../ImgForLaTexReport/labs5python/task7.py}
    \end{minipage}
\end{vvsu_figure}

\begin{vvsu_list}
    \item Создание словаря английский→русский
    \item Получение русского слова от пользователя
    \item Перебор пар ключ-значение в словаре
    \item Поиск совпадения с введенным словом
    \item Вывод английского перевода или сообщения об ошибке
\end{vvsu_list}

\subsection{Задание 8}
Создаем список вариантов, получаем выбор пользователя, генерируем случайный выбор компьютера, проверяем условия победы, выводим результат. На рисунке \ref{fig:task8} представлен полученный код.

\begin{vvsu_figure}{Листинг программы для задания 8}{fig:task8}
    \begin{minipage}{.75\textwidth}
        \lstinputlisting[language=Python,basicstyle=\fontsize{10}{10}\linespread{1}\selectfont\ttfamily]{../../ImgForLaTexReport/labs5python/task8.py}
    \end{minipage}
\end{vvsu_figure}

\begin{vvsu_list}
    \item Создание списка возможных вариантов
    \item Получение выбора пользователя
    \item Генерация случайного выбора компьютера
    \item Проверка всех условий победы
    \item Определение победителя или ничьи
    \item Вывод результата игры
\end{vvsu_list}

\newpage
\subsection{Задание 9}
Создаем список слов, создаем пустой словарь, перебираем слова, извлекаем первую букву, добавляем слово в список соответствующей буквы. На рисунке \ref{fig:task9} представлен полученный код.

\begin{vvsu_figure}{Листинг программы для задания 9}{fig:task9}
    \begin{minipage}{.75\textwidth}
        \lstinputlisting[language=Python,basicstyle=\fontsize{10}{10}\linespread{1}\selectfont\ttfamily]{../../ImgForLaTexReport/labs5python/task9.py}
    \end{minipage}
\end{vvsu_figure}

\begin{vvsu_list}
    \item Создание списка слов
    \item Создание пустого словаря для результата
    \item Перебор каждого слова из списка
    \item Извлечение первой буквы слова
    \item Проверка наличия буквы в словаре
    \item Добавление слова в список соответствующей буквы
    \item Формирование словаря {буква: [слова]}
\end{vvsu_list}

\newpage

\subsection{Задание 10}
Создаем список студентов с оценками, вычисляем средние оценки, находим студента с лучшим средним баллом, выводим результаты. На рисунке \ref{fig:task10} представлен \\
полученный код.


\begin{vvsu_figure}{Листинг программы для задания 10}{fig:task10}
    \begin{minipage}{.75\textwidth}
        \lstinputlisting[language=Python,basicstyle=\fontsize{10}{10}\linespread{1}\selectfont\ttfamily]{../../ImgForLaTexReport/labs5python/task10.py}
    \end{minipage}
\end{vvsu_figure}

\begin{vvsu_list}
    \item Создание списка студентов с оценками
    \item Вычисление средней оценки для каждого студента
    \item Запись средних оценок в словарь
    \item Поиск студента с максимальной средней оценкой
    \item Вывод словаря со средними оценками
    \item Вывод лучшего студента и его балла
\end{vvsu_list}

\end{document}