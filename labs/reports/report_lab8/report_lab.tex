\documentclass{vvsu}

\vvsuyear{2026}

\usepackage{graphicx}
\usepackage{tabularray}
\usepackage{siunitx}
\usepackage{listings}
\usepackage{amsmath}

\graphicspath{ {../ImgForLaTexReport/labs8python} }

\author{Герцов Д.Е.}

\input{listing_styles.tex}

\begin{document}

% Шапочка
\vvsuhead{\linespread{1}\selectfont{}МИНОБРНАУКИ РОССИИ\\
\vspace{10pt}Федеральное государственное бюджетное образовательное учреждение\\
высшего образования\\
\fontsize{13}{13}\selectfont{}<<ВЛАДИВОСТОКСКИЙ ГОСУДАРСТВЕННЫЙ УНИВЕРСИТЕТ>>\\
(ФГБОУ ВО <<ВВГУ>>)\\
\vspace{10pt}\fontsize{12}{12}\selectfont{}ИНСТИТУТ ИНФОРМАЦИОННЫХ ТЕХНОЛОГИЙ И АНАЛИЗА ДАННЫХ\\
КАФЕДРА ИНФОРМАЦИОННЫХ ТЕХНОЛОГИЙ И СИСТЕМ}

\title{Отчет\\по лабораторной работе №8}
\subtitle{по дисциплине\\<<Информатика и программирование>>}

\member{Студент\\ гр. БИН-25-3}{Герцов Д.Е.}
\member{Ассистент\\ преподавателя}{М.В. Водяницкий}

\maketitle

\toc

\begin{addition}{Задание}
    Реализовать консольную текстовую RPG--игру, демонстрирующую основные игровые механики:
    \begin{vvsu_list}
        \item Создание персонажа с выбором расы и случайными характеристиками
        \item Боевая система с уклонением, эффектами и уникальными способностями
        \item Инвентарь и экипировка
        \item Прокачка и распределение очков характеристик
        \item Исследование подземелья с развилками и случайными событиями
    \end{vvsu_list}
\end{addition}

\begin{introduction}

Развитие игровой индустрии на заре её становления было неразрывно связано с созданием текстовых RPG, которые демонстрировали возможности программирования для имитации сложных игровых миров.
Такие игры, запускаемые в консоли, требовали от разработчика глубокого понимания объектно--ориентированного программирования, работы с данными и проектирования игровых механик.
Целью данной лабораторной работы является реализация прототипа консольной текстовой RPG--игры, отражающего ключевые аспекты игрового процесса: создание персонажа с уникальными характеристиками, боевую систему с элементами случайности, управление инвентарём, прокачку и исследование процедурно генерируемого подземелья.
В ходе выполнения работы применяются принципы ООП, работа со случайными величинами, обработка пользовательского ввода и организация сложной логики взаимодействия между игровыми сущностями.
Актуальность работы обусловлена необходимостью закрепления навыков программирования на языке Python, а также понимания архитектурных решений, лежащих в основе даже самых простых игровых проектов.
Реализованный прототип может служить основой для дальнейшего расширения функционала и изучения более сложных игровых движков.
\end{introduction}

\section{Принципы проектирования}

При разработке RPG--игры были применены следующие принципы объектно--ориентированного проектирования:
Единственная ответственность (SRP): Каждый класс решает одну задачу. <<Character>> отвечает за характеристики и взаимодействие, <<Inventory>> -- за предметы, <<Enemy>> -- за поведение врагов.
Открытость/Закрытость (OCP): Система легко расширяема. Для добавления нового врага достаточно создать новый класс, наследуясь от <<Enemy>>, без изменения существующего кода.
Наследование: Все игровые сущности (игрок, враги) наследуются от базового класса <<Character>>, что обеспечивает единообразие интерфейса и повторное использование кода.
Инкапсуляция: Внутреннее состояние объектов (здоровье, урон, эффекты) скрыто, а доступ осуществляется через методы, что предотвращает некорректное изменение данных.
Модульность: Код разбит на логические модули (character, inventory, enemies, game mechanics), что упрощает поддержку, тестирование и понимание структуры проекта.

\section{Выполнение работы}

\subsection{Класс Character — инициализация и свойства}

Реализован базовый класс Character с инициализацией характеристик и вычисляемыми свойствами ИМТ урон защита уклонение

На рисунке \ref{fig:char1} представлен код

\begin{vvsu_figure}{Листинг character1.py}{fig:char1}
    \begin{minipage}{.95\textwidth}
        \lstinputlisting[language=Python,basicstyle=\fontsize{7}{8}\linespread{0.9}\selectfont\ttfamily]{../../ImgForLaTexReport/labs8python/character1.py}
    \end{minipage}
\end{vvsu_figure}

\begin{vvsu_list}
    \item Определены основные характеристики
    \item Реализованы свойства index mass total attack evasion chance
\end{vvsu_list}

Зачем это нужно: Базовый класс Character является фундаментом всей игровой системы. Он обеспечивает единый интерфейс для всех сущностей игрока и врагов что упрощает расширение и поддержку кода. Свойства такие как evasion chance позволяют реализовать сложную боевую механику без дублирования логики.

\subsection{Класс Character — методы взаимодействия}

Реализованы методы взаимодействия: получение урона, атака, лечение, прокачка, обработка эффектов

На рисунке \ref{fig:char2} -- код

\begin{vvsu_figure}{Листинг character2.py}{fig:char2}
    \begin{minipage}{.95\textwidth}
        \lstinputlisting[language=Python,basicstyle=\fontsize{7}{8}\linespread{0.9}\selectfont\ttfamily]{../../ImgForLaTexReport/labs8python/character2.py}
    \end{minipage}
\end{vvsu_figure}

\begin{vvsu_list}
    \item Методы take damage attack heal
    \item Система эффектов через update effects
    \item Прокачка через apply statpoints
\end{vvsu_list}

Зачем это нужно: Методы взаимодействия take damage attack реализуют основную игровую логику боя. Система эффектов позволяет добавлять временные состояния отравление кровотечение что значительно расширяет тактическую глубину игры. Прокачка через apply statpoints даёт игроку контроль над развитием персонажа.

\subsection{Инвентарь и экипировка}

Реализован класс <<Inventory>> для управления предметами

Поддерживается ограничение вместимости, экипировка оружия и брони, использование зелий

На рисунке \ref{fig:inventory} -- код

\begin{vvsu_figure}{Листинг inventory.py}{fig:inventory}
    \begin{minipage}{.95\textwidth}
        \lstinputlisting[language=Python,basicstyle=\fontsize{7}{8}\linespread{0.9}\selectfont\ttfamily]{../../ImgForLaTexReport/labs8python/inventory.py}
    \end{minipage}
\end{vvsu_figure}

Зачем это нужно: Класс <<Inventory>> управляет ресурсами игрока, что является ключевым элементом RPG--механик. Ограничение вместимости создаёт стратегический выбор: что взять, а что оставить. Экипировка напрямую влияет на боевые характеристики, связывая прогресс в исследовании с боевой мощью.

\subsection{Враги — базовые классы и первые типы}

Создан базовый класс <<Enemy>> и два уникальных врага: Пингвин кровотечение и Человек в маске воскрешение

На рисунке \ref{fig:enemy1} -- реализация

\begin{vvsu_figure}{Листинг enemy1.py}{fig:enemy1}
    \begin{minipage}{.95\textwidth}
        \lstinputlisting[language=Python,basicstyle=\fontsize{7}{8}\linespread{0.9}\selectfont\ttfamily]{../../ImgForLaTexReport/labs8python/enemy1.py}
    \end{minipage}
\end{vvsu_figure}

Зачем это нужно: Уникальные способности врагов кровотечение воскрешение ломают шаблонный бой и заставляют игрока адаптировать тактику. Это предотвращает монотонность и повышает вовлечённость. Базовый класс <<Enemy>> обеспечивает единообразие при добавлении новых типов противников.

\subsection{Враги — продвинутые типы и генерация}

Реализованы три уникальных врага и функция генерации с учётом этажа

На рисунке \ref{fig:enemy2} -- код

\begin{vvsu_figure}{Листинг enemy2.py}{fig:enemy2}
    \begin{minipage}{.95\textwidth}
        \lstinputlisting[language=Python,basicstyle=\fontsize{7}{8}\linespread{0.9}\selectfont\ttfamily]{../../ImgForLaTexReport/labs8python/enemy2.py}
    \end{minipage}
\end{vvsu_figure}

Зачем это нужно: Разнообразие врагов вампиризм усиление создаёт непредсказуемость и риск. Функция generate enemy с динамическим усилением по этажам гарантирует, что игра остаётся вызовом на протяжении всего прохождения, соответствующим принципу прогрессивной сложности.

\subsection{Расы — ядро уникальных механик}

Реализованы три уникальные расы с особыми механиками: Поглотитель впитывает урон, Гуль лечится при атаке, Пробуждённый усиливается в бою

На рисунке \ref{fig:races_core} -- код

\begin{vvsu_figure}{Листинг races core.py}{fig:races_core}
    \begin{minipage}{.95\textwidth}
        \lstinputlisting[language=Python,basicstyle=\fontsize{7}{8}\linespread{0.9}\selectfont\ttfamily]{../../ImgForLaTexReport/labs8python/races_core.py}
    \end{minipage}
\end{vvsu_figure}

Зачем это нужно: Уникальные расы с разными механиками дают игроку осмысленный выбор при старте. Случайная генерация характеристик в рамках расы обеспечивает реиграбельность: каждый запуск -- новый опыт.

\subsection{Создание персонажа}

Реализована логика выбора расы и генерации случайных характеристик в допустимых пределах

На рисунке \ref{fig:char_create} -- код

\begin{vvsu_figure}{Листинг character creation.py}{fig:char_create}
    \begin{minipage}{.95\textwidth}
        \lstinputlisting[language=Python,basicstyle=\fontsize{7}{8}\linespread{0.9}\selectfont\ttfamily]{../../ImgForLaTexReport/labs8python/character_creation.py}
    \end{minipage}
\end{vvsu_figure}

Зачем это нужно: Процесс создания персонажа задаёт начальные условия игры. Случайность в рамках расы обеспечивает баланс между предсказуемостью и вариативностью, что соответствует духу RPG.

\subsection{Основной игровой цикл}

Реализован главный цикл: исследование подземелья, выбор пути, обработка типов комнат бой, сундук, отдых

На рисунке \ref{fig:main_game} -- код

\begin{vvsu_figure}{Листинг main game.py}{fig:main_game}
    \begin{minipage}{.95\textwidth}
        \lstinputlisting[language=Python,basicstyle=\fontsize{7}{8}\linespread{0.9}\selectfont\ttfamily]{../../ImgForLaTexReport/labs8python/main_game1.py}
    \end{minipage}
\end{vvsu_figure}

Зачем это нужно: Основной цикл объединяет все игровые системы в единый процесс. Развилки с частичной видимостью создают напряжение и стратегическое планирование, что является основой исследования в RPG.

\subsection{Боевая система и механики}

Реализованы пошаговый бой, действия игрока атака, предмет, побег, обработка эффектов, генерация лута

На рисунке \ref{fig:game_mech} -- код

\begin{vvsu_figure}{Листинг game mechanics.py}{fig:game_mech}
    \begin{minipage}{.95\textwidth}
        \lstinputlisting[language=Python,basicstyle=\fontsize{7}{8}\linespread{0.9}\selectfont\ttfamily]{../../ImgForLaTexReport/labs8python/game_mechanics.py}
    \end{minipage}
\end{vvsu_figure}

Зачем это нужно: Боевая система -- сердце RPG. Пошаговые действия, уклонение, эффекты и добыча создают глубокую тактическую игру, где каждое решение имеет значение.

\section{Тестирование}
Тестирование текстовой RPG--игры проводилось с целью проверки корректности работы всех функциональных компонентов и обеспечения надежности игрового процесса. Были выполнены следующие виды тестирования:
- Попытка атаковать врага с последующей проверкой уклонения и расчёта урона;
- Тестирование боевой системы: оценка корректности применения эффектов (кровотечение, воскрешение) и работы уникальных способностей рас;
- Сохранение и восстановление состояния персонажа при прокачке и использовании предметов;
- Ручное распределение очков характеристик после повышения уровня с проверкой на некорректный ввод;
- Попытка использовать предметы из инвентаря при его переполнении и во время боя.
Все запуски прошли успешно, система работает корректно и стабильно.

\clearpage
\begin{conclusion}
В ходе выполнения лабораторной работы №8 был успешно реализован прототип консольной текстовой RPG--игры, полностью соответствующий поставленному заданию.
Программа демонстрирует все ключевые игровые механики: создание персонажа одной из трёх уникальных рас с генерацией случайных характеристик в рамках выбранной расы, исследование подземелья, состоящего из случайных комнат с частичной видимостью событий, пошаговую боевую систему с учётом уклонения и особых способностей, а также систему прокачки и управления инвентарём.
Были реализованы три расы персонажей -- Поглотитель, Гуль и Пробуждённый -- каждая из которых обладает уникальной игровой механикой, влияющей на тактику ведения боя.
Враги также наделены особыми способностями, такими как воскрешение, вампиризм и усиление в бою, что добавляет разнообразие и сложность игровому процессу.
Сложность игры динамически возрастает с каждым новым этажом подземелья.
Разработанное приложение представляет собой законченный, работоспособный продукт, написанный на языке Python с использованием принципов объектно--ориентированного программирования.
Код структурирован, читаем и легко поддаётся расширению.
Все поставленные задачи выполнены в полном объёме, что подтверждает успешное освоение материала по дисциплине Информатика и программирование.
\end{conclusion}

\end{document}